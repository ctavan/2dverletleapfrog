\documentclass[a4paper,11pt]{scrartcl}

\usepackage[utf8]{inputenc}
\usepackage[ngerman]{babel}

\usepackage{amsmath}
\usepackage{amsfonts}
\usepackage{amssymb}

\usepackage{enumerate}
\usepackage{xspace}

\usepackage[pdftex]{graphicx}
\usepackage[update]{epstopdf}
\usepackage{hyperref,breakurl}

\renewcommand{\vec}[1]{\mathbf{#1}}
\newcommand{\vr}{\vec{r}}
\newcommand{\vp}{\vec{p}}
\newcommand{\E}{\mathbf{e}}
\newcommand{\I}{\mathrm{i}}
\newcommand{\ind}[1]{_\mathrm{#1}}

\newcommand*{\ov}[2]{{{\vphantom{I}}^{#1}{\vphantom{I}}_{#2}}}
\newcommand*{\uv}[2]{{{\vphantom{I}}_{#1}{\vphantom{I}}^{#2}}}
\newcommand*{\Sr}[2][0]{\rule[#1ex]{0pt}{#2ex}}
\newcommand*{\zB}{z.\,B.\xspace }
\newcommand*{\dH}{d.\,h.\xspace }
\newcommand*{\iA}{i.\,a.\xspace }
\newcommand*{\uA}{u.\,a.\xspace}
\newcommand*{\vH}{\,\%\xspace}
\newcommand*{\dD}{\ensuremath{\mkern1mu\mathrm{d}}}
\newcommand*{\diff}[3][]{\frac{\dD^{#1}{#2}}{\dD^{}{#3}{}^{#1}}}
\newcommand*{\pdiff}[3][]{\frac{\partial^{#1}{#2}}{\partial^{}{#3}{}^{#1}}}
\newcommand*{\ArrF}[1]{\renewcommand{\arraystretch}{#1}}
\newcommand*{\Z}[1]{\ensuremath{\cdot10^{#1}}}
\newcommand*{\G}{\ensuremath{{}^{\circ}}\xspace}
\newcommand*{\GC}{\,\ensuremath{{}^{\circ}\text{C}}\xspace}
\newcommand*{\GF}{\,\ensuremath{{}^{\circ}\text{F}}\xspace}\let\sG\G % wg circ
\newcommand*{\entspricht}{\stackrel{\scriptscriptstyle\wedge}{=}}

\newcommand*{\nfrac}[2]{\nicefrac{#1}{#2}}

\renewcommand*{\bar}[1]{\overline{#1}}

\newcommand{\qed}{\hfill\ensuremath{\Box}}

\parindent0pt

% Title Page
% \title{Statistische Physik II - 3. Übungsblatt}
% \author{Dortje Schirok\\Christoph Tavan\\Patric Zimmermann}
% \date{9. Mai 2011}

\begin{document}

\section*{Aufgabe 15} % (fold)
\label{sec:aufgabe_15}
Wir betrachten einen Massepunkt im Zentralpotential \begin{align}
	\Phi(r) = - \frac{1}{r} \quad\text{mit}\quad r = \sqrt{x^2+y^2} .
\end{align}
Auf das Teilchen wirkt also eine Kraft \begin{align}
	\vec{F}(\vec{r}) = - \frac{\vec{r}}{r^3} = - \frac{1}{(x^2+y^2)^{3/2}} \left(
	\begin{array}{c}
	x
	\\ y\end{array} \right) .
\end{align}
Eine Kreisbahn (als Beispiel für eine geschlossene Bahn) erhält man wenn diese Kraft gerade die Zentripetalkraft kompensiert: \begin{align}
	\frac{m v^2}{r} \vec{e}_r &= \frac{r}{r^3} \vec{e}_r\\
	\Rightarrow\quad v &= \pm \frac{1}{\sqrt{m r}} 
\end{align}
Anfangsbedingungen, für die diese Bedingung erfüllt ist sind also \zB \begin{align}
	\vec{r}(t=0) &= \left(
	\begin{array}{c}
	r_0\\ 
	0\end{array} \right)\\
	\vec{v}(t=0) &= \left(
	\begin{array}{c}
	0\\ 
	v_0 \end{array} \right) \quad\text{mit}\quad v_0 = (m r_0)^{-1/2}.
\end{align}
Der Verlet-Algorithmus benötigt den Ort $\vec{r}(t=0)$ sowie den Ort im darauffolgenden Zeitschritt $\vec{r}(t=0+\delta t)$ als Anfangsbedingungen, wobei der zweite Ort anhand der Anfangsgeschwindigkeit ermittelt werden muss. Da wir eine Kreisbewegung fordern, fordern wir, dass der Abstand zum Ursprung erhalten bleibt. Für den Ort nach einem Zeitschritt ergibt sich dann näherungsweise \begin{align}
	\vec{r}(t=0+\delta t) &= \left(
	\begin{array}{c}
	\sqrt{r_0^2 - (v_0\,\delta t)^2 } \\ 
	v_0\,\delta t \end{array} \right).
\end{align}
In der Praxis muss der Zeitschritt $\delta t$ dann klein genug gewählt werden, damit diese Näherung wirklich auf kreisförmige Bahnen führt. In unserer Simulation musste für eine Bahn mit Radius 1 für den Verlet-Algorithmus $\delta t \sim 0.001$ gewählt werden.
% section aufgabe_15 (end)

\end{document}

